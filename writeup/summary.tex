\documentclass{article}

\usepackage{url}
\usepackage{code}

\title{Mutation Analysis of Smart Contracts}

\author{Alex Groce}

\begin{document}

\maketitle

\begin{abstract}
In mutation analysis (also called mutation testing), large numbers of small syntactic changes to a program are introduced, under the assumption that if the original program is correct, the modified program will be faulty, and the change will be the location of a fault.  These variations (called mutants) are considered ``killed'' if a test suite can distinguish them from the original, unmodified program, and surviving if it cannot.  Mutation score (percent killed mutants) is thus a measure, stronger than code coverage, of the fault detection power of a test suite.
Mutation analysis is typically used to evaluate the quality of a test suite, or perhaps a test-generation technique, and most often, at least until recently, only in an academic research setting.  This work proposes (1) use of \emph{differential} mutation result comparison to compare and improve static analysis tools, (2) use of statistics of such mutation scores to approximately compare the ``implicit specification'' strength of various programming languages, and (3) use of mutation analysis over combined static and dynamic analyses, combined with prioritization of surviving mutants, to strengthen smart contract testing and verification efforts.  We focus on smart contracts due to certain characteristics that are unusual or even unique to smart contract code, that make them especially suitable for a mutation-based approach.
\end{abstract}

\section{Introduction}

Mutation testing \cite{demillo1978hints,budd1980theoretical} uses small syntatic changes to a program to introduce synthetic ``faults,'' under the assumption that if the original version of a program is (mostly) correct, most small changes will therefore introduce a fault.  For the most part, mutation analysis has been used to evaluate test suites by computing a score (the number of mutants the suite detects, or ``kills'').  Most such use has been in research efforts, rather than practical testing efforts, though there has been sporadic use by interested developers.
In an ASE 2015 \cite{groce2015verified} paper and a 2018 journal extension \cite{groce2018verified} of that paper, Groce et al. proposed examining individual mutants that survive a formal verification or automated test generation process to detect and correct weaknesses in a specification or test generator.  The approach was able to expose bugs in a heavily-tested module of the Linux kernel \cite{mutKernel} and improve a heavily used test generator for the {\tt pyfakefs} file system.  Recently, mutation analysis has been adopted in industrial settings, though not for actual examination of all surviving mutants \cite{MutGoogle,ivankovic2018industrial}, a practice that is hard to scale to large code bodies.

\subsection{Why Smart Contracts?}

Mutation analysis and smart contracts are arguably a natural fit, for several reasons:

\begin{itemize}
\item Smart contract code (especially in Solidity) is usually relatively \emph{small}.  The explicit relationship between amount of computation performed, bytecode size, and deployment/execution cost in Ethereum introduces a strong incentive to minimize source code size.  Only two contracts in the etherscan data we examined had more than 3500 lines of code, and the largest contract had under 12KLOC.  The cost of mutation analysis, in terms of generating mutants, evaluating whether they are killed, and esepcially determining whether surviving mutants indicate weaknesses in testing, is directly proportional to code size, so smaller code makes mutation analysis more attractive.
\item Smart contract code is worth spending human and computational efforts, including mutation analysis efforts, in testing and verifying.  Because contracts may control considerably economic value, they are inherently high-criticality targets.  This also means that examining surviving mutants should be easier than in some other settings, since there is a strong motivation to produce effective test suites.
\item As noted below, it is likely that static methods can detect more bugs in smart contracts than in other code; the same pressures that generally keep LOC low keep contracts focused on core functionality, not irrelevant (and untested) bells and whistles.  Most code is doing something that can be checked, and in many cases the specification may be implicit in the almost-DSL like concentration of blockchain code; even if Solidity is Turing-complete, the range of things implemented in Solidity may be more restricted than for general-purpose programming languages.
\end{itemize}

\section{Differential Mutation Comparison}

We can say that a static analysis tool kills a mutant when the number of non-informational warnings or errors produced with respect to the code increases for the mutated version, compared to the original code.  This difference is most informative and easily interpreted when the original code produces no warnings or errors (it is ``clean''); in other cases, the tool may detect the mutant, but only change a previously generated message on the code in question, or on other code, due to the change, leading to an underestimate of the tool's effectiveness.

The value of the differential comparison lies in a few key points.  First, this is a measure that does not reward a tool that produces too many false positives.  The tool cannot simply flag all code as having a problem or it will perform poorly at the task of \emph{distinguishing} the mutated code from non-mutated (and presumably at least \emph{more} correct) code.  Based on verification and testing uses of mutations, it is safe to say that at least 50, and likely 60-70\% or more, of mutants that are not semantically equivalent to the original code are actually fault-inducing, so the task presented to a static analysis tool is the generalization of the task we ideally expect static analysis to perform:  to identify faulty code, without executing it, and, most critically, to distinguish faulty from correct code.  Obviously, many faults cannot be identified statically without a complete specification, or at low cost, but the measure of performance here is relative to other tools applied to the same code.  Second, and critically, this is an \emph{automatable} method that can provide an evaluation of static analysis tools over a large number of target source code files, without needing human effort to classify results as real bugs or false positives.  It is not clear that any other fully automatic method is competitively meaningful; it is possible that methods based on code changes from version control provide some of the same benefits, but these require classification of changes into bug-fixes and non-bug-fixes, and of course require version control history.  Also, history-based methods will be biased towards precisely those faults humans (or tools) were able to detect and fix.

This approach can also be used to improve tools.  If a tool fails to detect a mutant, but another tool does kill that mutant, the pattern involved can be examined to see if it offers a potential to improve the non-detecting tool.  In some cases, the change would produce too many false positives, but it is worth considering, especially when more than two tools are available to compare, and a majority do kill a mutant.

\subsection{Application to Solidity Static Analysis Tools}

We used the universalmutator tool \cite{universalmutator,regexpMut} to compare three Solidity static analysis tools using this differential approach: Trail of Bits' Slither \cite{slither}, Securify \cite{securify}, and SmartCheck \cite{smartcheck}.

\subsubsection{Solidity Compiler Version 0.5.4 Results}

{\bf Note that these results for Securify are likely invalid, due to ignoring {\tt Warning} level messages and only considering {\tt Violation} messages.}

We ran an initial comparison of the static analysis tools over a set of contracts that compile with Solidity version 0.5.4, the latest at the time of writing.  These contracts were taken from the {\bf Solidity by Example} webpage on the official Ethereum site, the {\bf Mastering Ethereum} book, and the (small) set of contracts in the etherscan results that compile with 0.5.4, for a total of 30 contracts, some of which are essentially the null contract.  These contracts produced 2,623 valid mutants to use in comparing the tools.  Slither performed better than the other tools for this set of contracts, with a mean mutation score of 0.11 over all contracts with any valid mutants, and 0.06 over the five contracts for which none of the tools found any issues.  By comparison, SmartCheck had means of 0.04 and 0.0 respectively for these sets, and Securify scores of 0.01 and 0.0.  The results for all contracts (but not for clean contracts only) are significant by Wilcoxon score.

In terms of cross-comparison on individual mutants, Slither also performed well.  It killed 265 mutants not detected by SmartCheck, and 294 mutants not killed by Securify.  In contrast, SmartCheck only detected 140 mutants not killed by Slither, and Securify only detected 40 mutants not detected by Slither.  Only 13 mutants were detected by both SmartCheck and Securify, but not by Slither, all of them involving a change of {\tt msg.sender} to {\tt tx.origin}.

\subsubsection{Solidity Compiler Version 0.4.24 Results}

An analysis of etherscan contracts that compile with {\tt solc} version 0.4.24 is in progress.  At present the 46,769 valid mutants generated by 100 randomly selected contracts are being analyzed.  This analysis is using a better definition of detection for Securify.  So far, Slither results are complete, with a mean score of 0.09 over all contracts and 0.11 over contracts that are clean with respect to Slither alone.  Analysis of 35 contracts with SmartCheck yields a mean score of 0.05 for all contracts, and 0.02 for contracts that are clean with respect to SmartCheck alone.

\section{Cross-Language Comparison}

The best Solidity analysis tool (Slither) detects 10\% or more of mutants in clean code.   Very preliminary results suggest that even well-engineered, widely-used tools in other languages (Python and C thus far) detect more in the range of 3-5\% of mutants.  As noted above, it is likely that the almost domain-specific nature of Solidity code, and the ability to thus use an implicit specification of behavior, the stronger type system (than C or Python), and the lack of difficult-to-analyze code such as UI, extensive data storage and manipulation, etc. makes Solidity code genuinely more suited to purely static analysis. 

\section{Full-Spectrum Mutation Analysis}

The practical 

\bibliographystyle{plain}
\bibliography{bibliography_ag,rahul}

\end{document}